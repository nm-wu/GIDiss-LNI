\documentclass{lni}

\pdfoptionpdfminorversion=5
% CW: to avoid warnings
%
\usepackage[ngerman]{babel}
\usepackage[T1]{fontenc}
% \usepackage[utf]{fontenc}
\usepackage[utf8]{inputenc}
\usepackage{fancyhdr}
\usepackage{changepage} %for changing topmargin on first page
% \usepackage{utf}
\usepackage{pdfpages}
\usepackage[pdftex,colorlinks=true,linkcolor=black,pdftitle="Ausgezeichnete Dissertationen 2019",pdfauthor="kile",pdfcreator="pdflatex",bookmarks=false]{hyperref}

%    \usepackage[cam,height=25cm]{crop} 
\fancypagestyle{fancyfirst}{
\lhead[\fancyplain{}{}]{\fancyplain{}{}}%
\rhead[\fancyplain{}{}]{\fancyplain{}{
\small Steffen  H{\"o}lldobler et. al. (Hrsg.): Ausgezeichnete
Informatikdissertationen 2019, \linebreak Lecture Notes in Informatics (LNI), Gesellschaft f{\"u}r Informatik, Bonn 2020 \hspace{5pt} \thepage \hspace{0.05cm}
}}%


% \fancyhead[RO]{}
} 

% \usepackage[ngerman]{babel}
% \usepackage{umlaute}
% ============================================================================
\makeatletter
\def\cleardoublepage{\clearpage\if@twoside \ifodd\c@page\else
\hbox{}
\vspace*{\fill}
\begin{center}
%This page intentionally contains only this sentence.
\end{center}
\vspace{\fill}
\thispagestyle{empty}
\newpage
\if@twocolumn\hbox{}\newpage\fi\fi\fi}
\makeatother
% ============================================================================
\newcounter{contribCounter}
%original stolen from self in tuned svmult-edition atmos04
% %parameters: 1:toc-title; 2:toc-authors; 3:running-title; 4:running-authors; 5:marker
% %Big Command, for inclusion of contributions, doing all sorts of peripheric stuff
% \renewcommand{\sheader}[5]{
% \cleardoublepage
% \addtocounter{contribCounter}{1}
% % \addcontentsline{toc}{chapter}{{\rm #2}\newline{\bfseries #1}\rm\dotfill}
% \addcontentsline{toc}{chapter}{{\bfseries \thecontribCounter~ #1}\newline{\rm #2}\rm\dotfill}
% \pdfbookmark[0]{\thecontribCounter~ #3}{#5}
% \pagestyle{fancy}
% \lhead[\fancyplain{}{\thepage}]{\fancyplain{}{\thecontribCounter~ #3}}
% \rhead[\fancyplain{}{#4}]{\fancyplain{}{\thepage}}
% \cfoot{}
% \thispagestyle{empty}
% \hypertarget{#5}{}
% }
%old parameters: 1:toc-title; 2:toc-authors; 3:running-title; 4:running-authors; 5:marker
%changed to %1: toc-title, 2: authors, 3: runningtitle
%Big Command, for inclusion of contributions, doing all sorts of peripheric stuff, also correct klick and point
\renewcommand{\sheader}[3]{
\cleardoublepage
\addtocontents{toc}{\protect\contentsline{section}{\protect\numberline{}%
{{\bfseries #2}\newline{\it #1}\rm\dotfill\ }}{\rm \pageref*{#2}}{#2}}
\hypertarget{#2}{}
\label{#2}
\pdfbookmark[0]{~~~#2}{#2}

\pagestyle{fancy}
\lhead[\fancyplain{}{\small \thepage \hspace{0.5em} #2}]{\fancyplain{}{}}
\rhead[\fancyplain{}{}]{\fancyplain{}{\small #3 \hspace{0.4em} \thepage}}
\cfoot{}
\thispagestyle{fancyfirst}
}



%%%% 
%%%% CW: Version of \sheader for names that can not be used as labels, e.g.
%%%% with \ss
%%%%
\newcommand{\sheaderxlabel}[4]{
\cleardoublepage
\addtocontents{toc}{\protect\contentsline{section}{\protect\numberline{}%
{{\bfseries #2}\newline{\it #1}\rm\dotfill\ }}{\rm \pageref*{#3}}{#3}}
\hypertarget{#3}{}
\label{#3}
\pdfbookmark[0]{~~~#2}{#3}

\pagestyle{fancy}
\lhead[\fancyplain{}{\small \thepage \hspace{0.5em} #2}]{\fancyplain{}{}}
\rhead[\fancyplain{}{}]{\fancyplain{}{\small #4 \hspace{0.4em} \thepage}}
% \lhead[\fancyplain{}{\thepage}]{\fancyplain{}{#4}}
% \rhead[\fancyplain{}{#2}]{\fancyplain{}{\thepage}}
\cfoot{}
\thispagestyle{fancyfirst}
}

\newcommand{\pdffile}[1]{%
\includepdf[pages=1-,noautoscale=true,pagecommand={},offset=7pt 42pt]{#1}%
}

\newcommand{\pdffilex}[1]{%
\includepdf[pages=1-,noautoscale=true,pagecommand={},offset=0pt 0pt]{#1}%
}

\newcommand{\pdffilexbrill}[1]{%
\includepdf[pages=1-,noautoscale=true,pagecommand={},offset=0pt -23pt]{#1}%
}

\newcommand{\pdffilexschaudt}[1]{%
\includepdf[pages=1-,noautoscale=true,pagecommand={},offset=15pt -25pt]{#1}%
}

\newcommand{\pdffilexlow}[1]{%
\includepdf[pages=1-,noautoscale=true,pagecommand={},offset=0pt -20pt]{#1}%
}

% ============================================================================

\begin{document}

\setcounter{page}{3}
\pagestyle{fancy}
\lhead[\fancyplain{}{\small\thepage}]{\fancyplain{}}
\rhead[\fancyplain{}]{\fancyplain{}{\small\thepage}}
\cfoot{}

\section*{Vorwort}
\pdfbookmark[1]{Vorwort}{Vorwort}

\newcommand{\Zitat}[1]{\glqq#1\grqq}

Die Gesellschaft für Informatik e.V.\ (GI) vergibt gemeinsam mit der Schweizer Informatik Gesellschaft (SI) und der Österreichischen Computergesellschaft (OCG) jährlich einen Preis für eine hervorragende Dissertation im Bereich der Informatik. Deutsche, österreichische und schweizer Universitäten und Hochschulen schlagen jeweils eine ausgezeichnet bewertete Dissertation vor, die zur Weiterentwicklung im Bereich Informatik und / oder in den Anwendungsgebieten beiträgt oder die Wechselwirkung zwischen Informatik und Gesellschaft untersucht. Somit sind die
im Auswahlverfahren vorgeschlagenen Kandidatinnen und Kandidaten bereits
\Zitat{Preisträger} ihrer Hochschule.

Die 25 Einreichungen für das Jahr 2019 belegen die Bedeutung des Dissertationspreises. Leider konnten wir in diesem Jahr aufgrund der Corona-Pandemie das Kolloquium zum Dissertationspreis nicht im Leibniz-Zentrum für Informatik Schloss Dagstuhl durchführen. Vielmehr haben wir online-Vorträge der Nominierten organisiert, die verteilt über mehrere Tage abliefen. Das wissenschaftliche Niveau der Vorträge war sehr hoch, die sich daran anschlie\ss{}enden Diskussionen aber leider doch zeitlich sehr beschränkt. 

Wie in jedem Jahr fiel es dem Nominierungsausschuss sehr schwer, eine Dissertation auszuwählen, die durch den Preis besonders gewürdigt wird. Mit
der Präsentation aller vorgeschlagenen Arbeiten in diesem Band wird die
Ungerechtigkeit, eine aus mehreren ebenbürtigen Dissertationen hervorzuheben,
etwas ausgeglichen. Der Band soll zudem einen Beitrag zum Wissenstransfer
innerhalb der Informatik und von den Universitäten und Hochschulen in die
Bereiche Technik, Wirtschaft und Gesellschaft leisten.

Die genannten Gesellschaften zeichnen Herrn Dr.~Jakub Tarnawski  für seine Arbeit \Zitat{New Graph Algorithms via Polyhedral Techniques} mit dem Dissertationspreis 2019 aus. \linebreak Dr.~Tarnawski hat zwei herausragende Ergebnisse erzielt: den ersten Approximationsalgorithmus f\"{u}r das asymmetrische Problem des Handlungsreisenden, der polynomielle Laufzeit aufweist und einen Approximationsfaktor garantiert, der unabh\"angig von der Anzahl der Knoten im Graph ist, und einen deterministischen parallelen Algorithmus f\"ur das perfekte Matching. 
Mit dieser Preisverleihung wird eine herausragende algorithmische Arbeit mit signifikanten methodischen  Innovationen gewürdigt. 


Ein gro\ss{}er Dank gilt dem Nominierungsausschuss für die sehr
effiziente und konstruktive Arbeit. Desweiteren möchte ich mich
besonders bedanken bei Frau Sylvia Wünsch für die Organisation der
online-Vorträge, Herrn Rüdiger Reischuk für die technischen
Vorbereitung und Durchführung der Vorträge, Herrn Dr.~Stefan Sobernig
und Frau Dr.~Lena Reinfelder für die Zusammenstellung des Bandes und
dem Team der Gesellschaft für Informatik e.V.\ für die technische
Unterstützung des Auswahlverfahrens.\bigskip

\noindent
Steffen Hölldobler\\
Dresden im November 2020

\vfill

% \centerline{\includegraphics[width=0.85\textwidth]{gidiss18gruppe.jpg}}

% % %%

\newpage

% \setcounter{page}{5}
\section*{Kandidat$^*$innen f\"ur den GI-Dissertationspreis 2019}

\setlength{\tabcolsep}{0pt}

\begin{tabular}{l@{\hspace{10pt}}l}
	Dr.\ Arp, Daniel & Technische Universität Braunschweig\\
	Dr.\ Baltes, Sebastian & The University Adelaide\\
	M.Sc.\ Bannach, Max & Universität zu Lübeck\\
	Dr.\ Behnke, Gregor & Albert-Ludwigs-Universität Freiburg\\
	Dr.-Ing.\ Bornschlegl, Marco Xaver & Fernuniversität Hagen\\
	Dr.\ Eisenbach, Markus & Technische Universität Ilmenau\\
	Dr.\ Engel, Heiko & Goethe-Universität Frankfurt a. M.\\
	Dr.\ Fillbrunn, Alexander & Universität Konstanz\\
	Dr.\ Goli, Mehran & Universität Bremen\\
	Dr.\ Goranci, Gramoz & Universität Wien\\
	Dr.-Ing.\ Hammernik, Kerstin & Technische Universität Graz\\
	Dr.-Ing.\ Hantke, Simone & Technische Universität München\\
	Dr.\ Kaminski, Benjamin & RWTH Aachen\\
	Dr. rer. nat.\ Kohl, Lisa Maria & Karlsruher Institut für Technologie\\
	Dr.\ Malavolta, Giulio & Friedrich-Alexander-Universität Erlangen-Nürnberg\\
	Dr.-Ing.\ Meuschke, Monique & Otto-von-Guericke-Universität Magdeburg\\
	Dr.\ Morris, Christopher & Technische Universität Dortmund\\
	Dr.\ Müller, Florian & Technische Universität Darmstadt\\
	Dr.-Ing.\ Pensel, Maximilian & Technische Universität Dresden\\
	Dr. rer. nat.\ Späth, Johannes & Universität Paderborn\\
	Dr.\ Tarnawski, Jakub & EPFL - Ecole Polytechnique Federale de Lausanne\\
	Dr.\ Wahl, Florian & Universität Passau\\
	Dr. rer. nat.\ Wingerath, Wolfram & Universität Hamburg\\
	Dr.\ Winkler, Kyrill & Technische Universität Wien\\
	Dr.\ Zulehner, Alwin & Johannis Kepler Universität Linz\\


\end{tabular}


%\vspace{8mm} 
\newpage
\section*{Mitglieder des Nominierungsausschusses\\ f\"ur den GI-Dissertationspreis 2019}
% ****

% \centerline{\includegraphics[width=\textwidth]{gidiss18jury}}
% \vspace{4pt}
% Von links nach rechts:
% \vspace{2pt}

\begin{tabular}{p{6.3cm}l}
  Prof.\ Dr.\ Steffen H\"olldobler (Vorsitzender) & Technische Universit\"at Dresden\\
  Prof.\ Dr.\ Sven Apel &  Universit{\"a}t des Saarlandes \\
  Prof.\ Dr.\ Abraham Bernstein & Universit\"at Z\"urich\\ 
  Prof.\ Dr.-Ing.\ Felix Freiling & Universit{\"a}t Erlangen-N{\"u}rnberg \\
  Prof.\ Dr.\ Hans-Peter Lenhof & Universit{\"a}t des Saarlandes\\
  Prof.\ Dr.\ Gustaf Neumann &   Wirtschaftsuniversit\"at Wien\\
  Prof.\ Dr.\ R{\"u}diger Reischuk & Universit{\"a}t zu L{\"u}beck \\
  Prof.\ Dr.\ Kay Uwe R{\"o}mer & TU Graz \\
  Prof.\ Dr.\ Bj{\"o}rn Scheuermann &  Humbold-Universit\"at zu Berlin \\
  Prof.\ Dr.\ Nicole Schweikardt & Humbold-Universit\"at zu Berlin\\
  Prof.\ Dr.\ Myra Spiliopoulou & Otto-von-Guericke-Universit{\"a}t Magdeburg\\
  Prof.\ Dr.\ Sabine S{\"u}sstrunk & {\'E}cole Polytechnique F{\'e}d{\'e}rale de Lausanne\\
  Prof.\ Dr.\ Klaus Wehrle & RWTH Aachen \\
\end{tabular}

\newpage

\pdfbookmark[1]{Inhalt}{Inhalt}

% \enlargethispage{-1\baselineskip}

\setlength{\parskip}{0.1cm}

\tableofcontents

% \newpage

% \end{document}

%offset : 1.: x-Achse(kleiner=links), y-Achse(groesser=oben)
%typisches pstopdf offset=-9pt -25pt meistens falsch ;) 	RICHTIG FUER ODD
%typisches pstopdf offset=9pt -25pt odd even problem ....	RICHTIG FUER EVEN
%offset=0pt -25pt,trim= 18pt 0pt 0pt 0pt			DAS IST DIE LOESUNG! (gammlig)
%typisches doc offset=-6pt 18pt

\sheader 
{Erkennung mobiler Schadsoftware mit maschinellen Lernverfahren}
{Arp, Daniel}
{Erkennung mobiler Schadsoftware mit maschinellen Lernverfahren}
\pdffile{pdfs/Arp-Daniel.pdf}

\sheader 
{Arbeitsgewohnheiten und Expertise von Softwareentwicklern}
{Baltes, Sebastian}
{Arbeitsgewohnheiten und Expertise von Softwareentwicklern}
\pdffile{pdfs/Baltes-Sebastian.pdf}

\sheader 
{Parallele Parametrisierte Algorithmen}
{Bannach, Max}
{Parallele Parametrisierte Algorithmen}
\pdffile{pdfs/Bannach-Max.pdf}

\sheader 
{Hierarchisches Planen durch Propositionale Logik}
{Behnke, Gregor}
{Hierarchisches Planen durch Propositionale Logik}
\pdffile{pdfs/Behnke-Gregor.pdf}

\sheader 
{Erweiterte Visuelle Benutzerschnittstellen für Big-Data-Analysen}
{Bornschlegl, Marco}
{Erweiterte Visuelle Benutzerschnittstellen für Big-Data-Analysen}
\pdffile{pdfs/Bornschlegl-Marco.pdf}

\sheader 
{Personenwiedererkennung mittels maschineller Lernverfahren}
{Eisenbach, Markus}
{Personenwiedererkennung mittels maschineller Lernverfahren}
\pdffile{pdfs/Eisenbach-Markus.pdf}

\sheader 
{Auslesekarte zur schnellen Verarbeitung von Detektordaten}
{Engel, Heiko}
{Auslesekarte zur schnellen Verarbeitung von Detektordaten}
\pdffile{pdfs/Engel-Heiko.pdf}

\sheader 
{Widening mit Hashbasierter Partitionierung des Hypothesenraums}
{Fillbrunn, Alexander}
{Widening mit Hashbasierter Partitionierung des Hypothesenraums}
\pdffile{pdfs/Fillbrunn-Alexander.pdf}

\sheader 
{Automatisierte Analyse virtueller Prototypen auf der ESL}
{Goli, Mehran}
{Automatisierte Analyse virtueller Prototypen auf der ESL}
\pdffile{pdfs/Goli-Mehran.pdf}

\sheader 
{Algorithmen für graphbasierte dynamische Probleme}
{Goranci, Gramoz}
{Algorithmen für graphbasierte dynamische Probleme}
\pdffile{pdfs/Goranci-Gramoz.pdf}

\sheader 
{Variationsnetzwerke f{\"u}r die medizinische Bildrekonstruktion}
{Hammernik, Kerstin}
{Variationsnetzwerke f{\"u}r die medizinische Bildrekonstruktion}
\pdffile{pdfs/Hammernik-Kerstin.pdf}

\sheader 
{Intelligentes, spielorientiertes Crowdsourcing f{\"u}r Audioverarbeitung}
{Hantke, Simone}
{Intelligentes, spielorientiertes Crowdsourcing f{\"u}r Audioverarbeitung}
\pdffile{pdfs/Hantke-Simone.pdf}

\sheader 
{Erweiterte Kalk{\"u}le schw{\"a}chster Vorbedingungen f{\"u}r probabilistische Programme}
{Kaminski, Benjamin}
{Erweiterte Kalk{\"u}le schw{\"a}chster Vorbedingungen f{\"u}r probabilistische Programme}
\pdffile{pdfs/Kaminski-Benjamin.pdf}


\sheader 
{Verbesserte Kommunikationskomplexit{\"a}t in der Kryptographie}
{Kohl, Lisa}
{Verbesserte Kommunikationskomplexit{\"a}t in der Kryptographie}
\pdffile{pdfs/Kohl-Lisa.pdf}

\sheader 
{Kryptographische Uhren und Anwendungen}
{Malavolta, Giulio}
{Kryptographische Uhren und Anwendungen}
\pdffile{pdfs/Malavolta-Giulio.pdf}

\sheader 
{Visuelle Risikoanalyse und Therapieplanung zerebraler Aneurysmen}
{Meuschke, Monique}
{Visuelle Risikoanalyse und Therapieplanung zerebraler Aneurysmen}
\pdffile{pdfs/Meuschke-Monique.pdf}

\sheader 
{Lernen mit Graphen: Kern- und neuronale Methoden}
{Morris, Christopher}
{Lernen mit Graphen: Kern- und neuronale Methoden}
\pdffile{pdfs/Morris-Christopher.pdf}

\sheader 
{Around-Body Interaction}
{M{\"u}ller, Florian}
{Around-Body Interaction}
\pdffile{pdfs/Mueller-Florian.pdf}

\sheader 
{Schlussfolgern in Defeasible Beschreibungslogiken}
{Pensel, Maximilian}
{Schlussfolgern in Defeasible Beschreibungslogiken}
\pdffile{pdfs/Pensel-Maximilian.pdf}

\sheader 
{Synchronisierte Pushdown Systeme für Datenfluss-Analysen}
{Sp{\"a}th, Johannes}
{Synchronisierte Pushdown Systeme für Datenfluss-Analysen}
\pdffile{pdfs/Spaeth-Johannes.pdf}

\sheader 
{Neue Graphen-Algorithmen mittels polyedrischer Methoden}
{Tarnawski, Jakub}
{Neue Graphen-Algorithmen mittels polyedrischer Methoden}
\pdffile{pdfs/Tarnawski-Jakub.pdf}

\sheader 
{Methoden zum Monitoring des zirkadianen Rhythmus im Alltag}
{Wahl, Florian}
{Methoden zum Monitoring des zirkadianen Rhythmus im Alltag}
\pdffile{pdfs/Wahl-Florian.pdf}

\sheader 
{Skalierbare Push-basierte Echtzeitanfragen für Pull-basierte DBs}
{Wingerath, Wolfram}
{Skalierbare Push-basierte Echtzeitanfragen für Pull-basierte DBs}
\pdffile{pdfs/Wingerath-Wolfram.pdf}

\sheader 
{L{\"o}sbarkeit von Consensus unter Nachrichtengegnern}
{Winkler, Kyrill}
{L{\"o}sbarkeit von Consensus unter Nachrichtengegnern}
\pdffile{pdfs/Winkler-Kyrill.pdf}

\sheader 
{Entwurfsautomatisierung f{\"u}r Quantencomputer}
{Zulehner, Alwin}
{Entwurfsautomatisierung f{\"u}r Quantencomputer}
\pdffile{pdfs/Zulehner-Alwin.pdf}

\end{document}
