\documentclass{lni}

\pdfoptionpdfminorversion=5
% CW: to avoid warnings
%
\usepackage[ngerman]{babel}
\usepackage[T1]{fontenc}
% \usepackage[utf]{fontenc}
\usepackage[utf8]{inputenc}
\usepackage{fancyhdr}
\usepackage{changepage} %for changing topmargin on first page
% \usepackage{utf}
\usepackage{pdfpages}
\usepackage[pdftex,colorlinks=true,linkcolor=black,pdftitle="Ausgezeichnete Dissertationen 2019",pdfauthor="kile",pdfcreator="pdflatex",bookmarks=false]{hyperref}

%    \usepackage[cam,height=25cm]{crop} 
\fancypagestyle{fancyfirst}{
\lhead[\fancyplain{}{}]{\fancyplain{}{}}%
\rhead[\fancyplain{}{}]{\fancyplain{}{
\small Steffen  H{\"o}lldobler et. al. (Hrsg.): Ausgezeichnete
Informatikdissertationen 2019, \linebreak Lecture Notes in Informatics (LNI), Gesellschaft f{\"u}r Informatik, Bonn 2020 \hspace{5pt} \thepage \hspace{0.05cm}
}}%


% \fancyhead[RO]{}
} 

% \usepackage[ngerman]{babel}
% \usepackage{umlaute}
% ============================================================================
\makeatletter
\def\cleardoublepage{\clearpage\if@twoside \ifodd\c@page\else
\hbox{}
\vspace*{\fill}
\begin{center}
%This page intentionally contains only this sentence.
\end{center}
\vspace{\fill}
\thispagestyle{empty}
\newpage
\if@twocolumn\hbox{}\newpage\fi\fi\fi}
\makeatother
% ============================================================================
\newcounter{contribCounter}
%original stolen from self in tuned svmult-edition atmos04
% %parameters: 1:toc-title; 2:toc-authors; 3:running-title; 4:running-authors; 5:marker
% %Big Command, for inclusion of contributions, doing all sorts of peripheric stuff
% \renewcommand{\sheader}[5]{
% \cleardoublepage
% \addtocounter{contribCounter}{1}
% % \addcontentsline{toc}{chapter}{{\rm #2}\newline{\bfseries #1}\rm\dotfill}
% \addcontentsline{toc}{chapter}{{\bfseries \thecontribCounter~ #1}\newline{\rm #2}\rm\dotfill}
% \pdfbookmark[0]{\thecontribCounter~ #3}{#5}
% \pagestyle{fancy}
% \lhead[\fancyplain{}{\thepage}]{\fancyplain{}{\thecontribCounter~ #3}}
% \rhead[\fancyplain{}{#4}]{\fancyplain{}{\thepage}}
% \cfoot{}
% \thispagestyle{empty}
% \hypertarget{#5}{}
% }
%old parameters: 1:toc-title; 2:toc-authors; 3:running-title; 4:running-authors; 5:marker
%changed to %1: toc-title, 2: authors, 3: runningtitle
%Big Command, for inclusion of contributions, doing all sorts of peripheric stuff, also correct klick and point
\renewcommand{\sheader}[3]{
\cleardoublepage
\addtocontents{toc}{\protect\contentsline{section}{\protect\numberline{}%
{{\bfseries #2}\newline{\it #1}\rm\dotfill\ }}{\rm \pageref*{#2}}{#2}}
\hypertarget{#2}{}
\label{#2}
\pdfbookmark[0]{~~~#2}{#2}

\pagestyle{fancy}
\lhead[\fancyplain{}{\small \thepage \hspace{0.5em} #2}]{\fancyplain{}{}}
\rhead[\fancyplain{}{}]{\fancyplain{}{\small #3 \hspace{0.4em} \thepage}}
\cfoot{}
\thispagestyle{fancyfirst}
}



%%%% 
%%%% CW: Version of \sheader for names that can not be used as labels, e.g.
%%%% with \ss
%%%%
\newcommand{\sheaderxlabel}[4]{
\cleardoublepage
\addtocontents{toc}{\protect\contentsline{section}{\protect\numberline{}%
{{\bfseries #2}\newline{\it #1}\rm\dotfill\ }}{\rm \pageref*{#3}}{#3}}
\hypertarget{#3}{}
\label{#3}
\pdfbookmark[0]{~~~#2}{#3}

\pagestyle{fancy}
\lhead[\fancyplain{}{\small \thepage \hspace{0.5em} #2}]{\fancyplain{}{}}
\rhead[\fancyplain{}{}]{\fancyplain{}{\small #4 \hspace{0.4em} \thepage}}
% \lhead[\fancyplain{}{\thepage}]{\fancyplain{}{#4}}
% \rhead[\fancyplain{}{#2}]{\fancyplain{}{\thepage}}
\cfoot{}
\thispagestyle{fancyfirst}
}

\newcommand{\pdffile}[1]{%
\includepdf[pages=1-,noautoscale=true,pagecommand={},offset=7pt 42pt]{#1}%
}

\newcommand{\pdffilex}[1]{%
\includepdf[pages=1-,noautoscale=true,pagecommand={},offset=0pt 0pt]{#1}%
}

\newcommand{\pdffilexbrill}[1]{%
\includepdf[pages=1-,noautoscale=true,pagecommand={},offset=0pt -23pt]{#1}%
}

\newcommand{\pdffilexschaudt}[1]{%
\includepdf[pages=1-,noautoscale=true,pagecommand={},offset=15pt -25pt]{#1}%
}

\newcommand{\pdffilexlow}[1]{%
\includepdf[pages=1-,noautoscale=true,pagecommand={},offset=0pt -20pt]{#1}%
}

% ============================================================================

\begin{document}

\setcounter{page}{3}
\pagestyle{fancy}
\lhead[\fancyplain{}{\small\thepage}]{\fancyplain{}}
\rhead[\fancyplain{}]{\fancyplain{}{\small\thepage}}
\cfoot{}

\section*{Vorwort}
\pdfbookmark[1]{Vorwort}{Vorwort}

\newcommand{\Zitat}[1]{\glqq#1\grqq}

Die Gesellschaft für Informatik e.V.\ (GI) vergibt gemeinsam mit der Schweizer Informatik Gesellschaft (SI) und der Österreichischen Computergesellschaft (OCG) jährlich einen Preis für eine hervorragende Dissertation im Bereich der Informatik. Deutsche, österreichische und schweizer Universitäten und Hochschulen schlagen jeweils eine ausgezeichnet bewertete Dissertation vor, die zur Weiterentwicklung im Bereich Informatik und / oder in den Anwendungsgebieten beiträgt oder die Wechselwirkung zwischen Informatik und Gesellschaft untersucht. Somit sind die
im Auswahlverfahren vorgeschlagenen Kandidatinnen und Kandidaten bereits
\Zitat{Preisträger} ihrer Hochschule.

Die 25 Einreichungen für das Jahr 2019 belegen die Bedeutung des Dissertationspreises. Leider konnten wir in diesem Jahr aufgrund der Corona-Pandemie das Kolloquium zum Dissertationspreis nicht im Leibniz-Zentrum für Informatik Schloss Dagstuhl durchführen. Vielmehr haben wir online-Vorträge der Nominierten organisiert, die verteilt über mehrere Tage abliefen. Das wissenschaftliche Niveau der Vorträge war sehr hoch, die sich daran anschlie\ss{}enden Diskussionen aber leider doch zeitlich sehr beschränkt. 

Wie in jedem Jahr fiel es dem Nominierungsausschuss sehr schwer, eine Dissertation auszuwählen, die durch den Preis besonders gewürdigt wird. Mit
der Präsentation aller vorgeschlagenen Arbeiten in diesem Band wird die
Ungerechtigkeit, eine aus mehreren ebenbürtigen Dissertationen hervorzuheben,
etwas ausgeglichen. Der Band soll zudem einen Beitrag zum Wissenstransfer
innerhalb der Informatik und von den Universitäten und Hochschulen in die
Bereiche Technik, Wirtschaft und Gesellschaft leisten.

Die genannten Gesellschaften zeichnen Herrn Dr.~Jakub Tarnawski  für seine Arbeit \Zitat{New Graph Algorithms via Polyhedral Techniques} mit dem Dissertationspreis 2019 aus. \linebreak Dr.~Tarnawski hat zwei herausragende Ergebnisse erzielt: den ersten Approximationsalgorithmus f\"{u}r das asymmetrische Problem des Handlungsreisenden, der polynomielle Laufzeit aufweist und einen Approximationsfaktor garantiert, der unabh\"angig von der Anzahl der Knoten im Graph ist, und einen deterministischen parallelen Algorithmus f\"ur das perfekte Matching. 
Mit dieser Preisverleihung wird eine herausragende algorithmische Arbeit mit signifikanten methodischen  Innovationen gewürdigt. 


Ein gro\ss{}er Dank gilt dem Nominierungsausschuss für die sehr
effiziente und konstruktive Arbeit. Desweiteren möchte ich mich
besonders bedanken bei Frau Sylvia Wünsch für die Organisation der
online-Vorträge, Herrn Rüdiger Reischuk für die technischen
Vorbereitung und Durchführung der Vorträge, Herrn Dr.~Stefan Sobernig
und Frau Dr.~Lena Reinfelder für die Zusammenstellung des Bandes und
dem Team der Gesellschaft für Informatik e.V.\ für die technische
Unterstützung des Auswahlverfahrens.\bigskip

\noindent
Steffen Hölldobler\\
Dresden im November 2020

\vfill

% \centerline{\includegraphics[width=0.85\textwidth]{gidiss18gruppe.jpg}}

% % %%

\newpage

% \setcounter{page}{5}
\section*{Kandidat$^*$innen f\"ur den GI-Dissertationspreis 2019}

\setlength{\tabcolsep}{0pt}

\begin{tabular}{l@{\hspace{10pt}}l}
& \\
\end{tabular}


%\vspace{8mm} 
\newpage
\section*{Mitglieder des Nominierungsausschusses\\ f\"ur den GI-Dissertationspreis 2019}
% ****

% \centerline{\includegraphics[width=\textwidth]{gidiss18jury}}
% \vspace{4pt}
% Von links nach rechts:
% \vspace{2pt}

\begin{tabular}{p{6.3cm}l}
  Prof.\ Dr.\ Steffen H\"olldobler (Vorsitzender) & Technische Universit\"at Dresden\\
  Prof.\ Dr.\ Sven Apel &  Universit{\"a}t des Saarlandes \\
  Prof.\ Dr.\ Abraham Bernstein & Universit\"at Z\"urich\\ 
  Prof.\ Dr.-Ing.\ Felix Freiling & Universit{\"a}t Erlangen-N{\"u}rnberg \\
  Prof.\ Dr.\ Hans-Peter Lenhof & Universit{\"a}t des Saarlandes\\
  Prof.\ Dr.\ Gustaf Neumann &   Wirtschaftsuniversit\"at Wien\\
  Prof.\ Dr.\ R{\"u}diger Reischuk & Universit{\"a}t zu L{\"u}beck \\
  Prof.\ Dr.\ Kay Uwe R{\"o}mer & TU Graz \\
  Prof.\ Dr.\ Bj{\"o}rn Scheuermann &  Humbold-Universit\"at zu Berlin \\
  Prof.\ Dr.\ Nicole Schweikardt & Humbold-Universit\"at zu Berlin\\
  Prof.\ Dr.\ Myra Spiliopoulou & Otto-von-Guericke-Universit{\"a}t Magdeburg\\
  Prof.\ Dr.\ Sabine S{\"u}sstrunk & {\'E}cole Polytechnique F{\'e}d{\'e}rale de Lausanne\\
  Prof.\ Dr.\ Klaus Wehrle & RWTH Aachen \\
\end{tabular}

\newpage

\pdfbookmark[1]{Inhalt}{Inhalt}

% \enlargethispage{-1\baselineskip}

\setlength{\parskip}{0.1cm}

\tableofcontents

% \newpage

% \end{document}

%offset : 1.: x-Achse(kleiner=links), y-Achse(groesser=oben)
%typisches pstopdf offset=-9pt -25pt meistens falsch ;) 	RICHTIG FUER ODD
%typisches pstopdf offset=9pt -25pt odd even problem ....	RICHTIG FUER EVEN
%offset=0pt -25pt,trim= 18pt 0pt 0pt 0pt			DAS IST DIE LOESUNG! (gammlig)
%typisches doc offset=-6pt 18pt

\sheader 
{Personalized Recommender Systems for Software Product Line Configurations}
% {Personalisierte Recommender Systems f{\"u}r Software-Produktlinienkonfigurationen}
{Arriel, Juliana}
{Personalized Recommender Systems for Software Product Line Configurations}
% {Personalisierte Recommender Systems f{\"u}r Software-Produktlinienkonfigurationen}
% \pdffile{pdfs/invited_paper_11}


% \sheader 
% {Quantifying and Mitigating Privacy Risks in Biomedical Data}
% % {Quantifizierung und Schutz der Privatsph{\"a}re in der (Epi-)genetik}
% {Berrang, Pascal}
% {Quantifying and Mitigating Privacy Risks in Biomedical Data}
% % {Quantifizierung und Schutz der Privatsph{\"a}re in der (Epi-)genetik}
% \pdffile{pdfs/invited_paper_3}


% \sheader 
% {Behavior Verification for Business Processes \\ based on Testing and Anomaly Detection} %Testing and Anomaly Detection for Business Processes
% % {Verhaltensverifizierung f{\"u}r Gesch{\"a}ftsprozesse basierend auf Testverfahren und Anomalieerkennung}
% {B{\"o}hmer, Kristof}
% {Behavior Verification for Business Processes based on Testing and Anomaly Detection}
% % {Verhaltensverifizierung f{\"u}r Gesch{\"a}ftsprozesse basierend auf Testverfahren und Anomalieerkennung}
% \pdffile{pdfs/invited_paper_14}

% \sheader 
% {Online Learning of Latent Linguistic Structure with Approximate Search}
% % {Inkrementelles Lernen latenter linguistischer Strukturen durch approximative Suche}
% {Bj{\"o}rkelund, Anders}
% {Online Learning of Latent Linguistic Structure with Approximate Search}
% % {Inkrementelles Lernen latenter linguistischer Strukturen durch approximative Suche}
% \pdffile{pdfs/invited_paper_20}


% \sheader 
% {A Physical Layer Experimentation Framework for Automotive WLAN}
% % {Eine Experimentierumgebung zur Studie der physikalischen
% % Schicht einer WLAN-Variante für die Anwendung in
% % Fahrzeugnetzen}
% {Bloessl, Bastian}
% % {Eine Experimentierumgebung zur Studie der physikalischen
% % Schicht einer WLAN-Variante für die Anwendung in
% % Fahrzeugnetzen}
% {A Physical Layer Experimentation Framework for Automotive WLAN}
% \pdffile{pdfs/invited_paper_18}


% \sheader 
% {Learning to Predict Dense Correspondences for 6D Pose Estimation}
% % {6D Posensch{\"a}tzung mit gelernten, dichten
% % Korrespondenzvorhersagen}
% {Brachmann, Eric}
% {Learning to Predict Dense Correspondences for 6D Pose Estimation}
% % {6D Posensch{\"a}tzung mit gelernten, dichten
% % Korrespondenzvorhersagen}
% \pdffile{pdfs/invited_paper_9}


% \sheader 
% {Behaviour-Aware Mobile Touch Interfaces}
% % {Verhaltenssensitive Nutzerschnittstellen für mobile Ger{\"a}te
% % mit ber{\"u}hrungsempfindlichem Bildschirm}
% {Buschek, Daniel}
% {Behaviour-Aware Mobile Touch Interfaces}
% % {Verhaltenssensitive Nutzerschnittstellen für mobile Ger{\"a}te
% % mit ber{\"u}hrungsempfindlichem Bildschirm}
% \pdffile{pdfs/invited_paper_10}


% \sheader 
% {Computational Methods for Small Molecule Identification}
% % {Die Analyse kleiner Molek{\"u}le mittels Methoden der
% % Kombinatorik und des maschinellen Lernens}
% {D{\"u}hrkop, Kai}
% {Computational Methods for Small Molecule Identification}
% % {Die Analyse kleiner Molek{\"u}le mittels Methoden der
% % Kombinatorik und des maschinellen Lernens}
% \pdffile{pdfs/invited_paper_7}


% \sheader 
% {Differential Cryptanalysis of Symmetric Primitives}
% {Eichlseder, Maria}
% {Differential Cryptanalysis of Symmetric Primitives}
% \pdffile{pdfs/invited_paper_22}


% \sheader 
% {Quadrator Control for Accurate Agile Flight}
% {Faessler, Matthias}
% {Quadrator Control for Accurate Agile Flight}
% % \pdffile{pdfs/invited_paper_}


% \sheader 
% {Programming Models and Extensive Evaluation Support\\ for MPTCP Scheduling, Adaptation Decisions, and DASH Video Streaming} % zu lang
% {Fr{\"o}mmmgen, Alexander}
% {Programmiermodelle \& Unterstützung f{\"u}r umfassende Evaluationen für Kommunikationssyst}
% \pdffile{pdfs/invited_paper_13}


% \sheader 
% {Methods and Benchmarks for Auto-Scaling Mechanisms\\ in Elastic Cloud Environments}
% {Herbst, Nikolas}
% {Methods and Benchmarks for Auto-Scaling Mechanisms in Elastic Cloud Environments}
% \pdffile{pdfs/invited_paper_6}


% \sheader 
% {A Verified ODE Solver and Smale's 14th Problem}
% {Immler, Fabian}
% {A Verified ODE Solver and Smale's 14th Problem}
% \pdffile{pdfs/invited_paper_24}


% \sheader 
% {Statistical appearance models based \\on probabilistic correspondences for medical image analysis} % zu lang
% % {Statistische Appearance-Modelle basierend auf probabilistischen Korrespondenzen f{\"u}r die medizinische Bildanalyse}
% {Kr{\"u}ger, Julia}
% {Probabilistische Appearance-Modelle f{\"u}r die medizinische Bildanalyse}
% % {Statistische Appearance-Modelle basierend auf probabilistischen Korrespondenzen f{\"u}r die medizinische Bildanalyse}
% \pdffile{pdfs/invited_paper_26}


% \sheader 
% {The Power of Locality Exploring the Limits \\of Randomness in Distributed Computing}
% {Maus, Yannic}
% {The Power of Locality Exploring the Limits of Randomness in Distributed Computing}
% \pdffile{pdfs/invited_paper_12}


% \sheader 
% {Efficient Learning from Comparisons}
% {Maystre, Lucas}
% {Efficient Learning from Comparisons}
% \pdffile{pdfs/invited_paper_15}


% \sheader 
% {Behavior-Based Architecture Conformance Checking}
% {Nicolaescu, Ana}
% {Behavior-Based Architecture Conformance Checking}
% \pdffile{pdfs/invited_paper_17}


% \sheader 
% {Multitarget Tracking Using Orientation Estimation for Optical Belt Sorting}
% {Pfaff, Florian}
% {Multitarget Tracking Using Orientation Estimation for Optical Belt Sorting}
% \pdffile{pdfs/invited_paper_16}


% \sheader 
% {Exponential families on resource-constrained systems}
% {Piatkowski, Nico Philipp}
% {Exponential families on resource-constrained systems}
% \pdffile{pdfs/invited_paper_27}


% \sheader 
% {Robust Speech Enhancement Using\\ Statistical Signal Processing and Machine-Learning}
% {Rehr, Robert}
% {Robust Speech Enhancement Using Statistical Signal Processing and Machine-Learning}
% \pdffile{pdfs/invited_paper_21}


% \sheader 
% {Memory-safe Execution of Low-level Languages on a Java Virtual Machine}
% {Rigger, Manuel}
% {Memory-safe Execution of Low-level Languages on a Java Virtual Machine}
% \pdffile{pdfs/invited_paper_8}


% \sheader 
% {Knowledge Generation in Visual Analytics: \\Integrating Human and Machine Intelligence for Exploration of Big Data} % zu lang
% {Sacha, Dominik}
% {Wissensbildung in der visuellen Datenanalyse}
% \pdffile{pdfs/invited_paper_2}


% \sheader 
% {Petri Net Synthesis and Modal Specifications}
% {Schlachter, Uli Christian}
% {Petri Net Synthesis and Modal Specifications}
% \pdffile{pdfs/invited_paper_25}


% \sheader 
% {Learning Representations of Atomistic Systems with Deep Neural Networks}
% {Sch{\"u}tt, Kristof}
% {Learning Representations of Atomistic Systems with Deep Neural Networks}
% \pdffile{pdfs/invited_paper_4}


% \sheader 
% {Synthesis and Optimization \\for Logic-in-Memory Computing using Memristive Devices}
% {Shirinzadeh, Saeideh}
% {Synthesis and Optimization for Logic-in-Memory Computing using Memristive Devices}
% \pdffile{pdfs/invited_paper_1}


% \sheader 
% {Evaluating Experiences of Autistic Children with Technologies in Co-Design}
% {Spiel, Katta}
% {Evaluating Experiences of Autistic Children with Technologies in Co-Design}
% \pdffile{pdfs/invited_paper_19}


% \sheader 
% {A Categorical Approach to Algebraic}
% {Urbat, Henning}
% {A Categorical Approach to Algebraic}
% \pdffile{pdfs/invited_paper_23}


% \sheader 
% {Misbehavior Detection in Cooperative Intelligent Transport Systems}
% {van der Heijden, Rens}
% {Misbehavior Detection in Cooperative Intelligent Transport Systems}
% \pdffile{pdfs/invited_paper_5}

\end{document}
