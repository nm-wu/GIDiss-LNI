\documentclass{lni}

\pdfoptionpdfminorversion=5
% CW: to avoid warnings
%
\usepackage[ngerman]{babel}
\usepackage[T1]{fontenc}
% \usepackage[utf]{fontenc}
\usepackage[utf8]{inputenc}
\usepackage{fancyhdr}
\usepackage{changepage} %for changing topmargin on first page
% \usepackage{utf}
\usepackage{pdfpages}
\usepackage[pdftex,colorlinks=true,linkcolor=black,pdftitle="Ausgezeichnete Dissertationen 2018",pdfauthor="kile",pdfcreator="pdflatex",bookmarks=false]{hyperref}

%    \usepackage[cam,height=25cm]{crop} 
\fancypagestyle{fancyfirst}{
\lhead[\fancyplain{}{}]{\fancyplain{}{}}%
\rhead[\fancyplain{}{}]{\fancyplain{}{
\small Steffen  H{\"o}lldobler et. al. (Hrsg.): Ausgezeichnete Informatikdissertationen 2018, \linebreak Lecture Notes in Informatics (LNI), Gesellschaft f{\"u}r Informatik, Bonn 2019 \hspace{5pt} \thepage \hspace{0.05cm}
}}%


% \fancyhead[RO]{}
} 

% \usepackage[ngerman]{babel}
% \usepackage{umlaute}
% ============================================================================
\makeatletter
\def\cleardoublepage{\clearpage\if@twoside \ifodd\c@page\else
\hbox{}
\vspace*{\fill}
\begin{center}
%This page intentionally contains only this sentence.
\end{center}
\vspace{\fill}
\thispagestyle{empty}
\newpage
\if@twocolumn\hbox{}\newpage\fi\fi\fi}
\makeatother
% ============================================================================
\newcounter{contribCounter}
%original stolen from self in tuned svmult-edition atmos04
% %parameters: 1:toc-title; 2:toc-authors; 3:running-title; 4:running-authors; 5:marker
% %Big Command, for inclusion of contributions, doing all sorts of peripheric stuff
% \renewcommand{\sheader}[5]{
% \cleardoublepage
% \addtocounter{contribCounter}{1}
% % \addcontentsline{toc}{chapter}{{\rm #2}\newline{\bfseries #1}\rm\dotfill}
% \addcontentsline{toc}{chapter}{{\bfseries \thecontribCounter~ #1}\newline{\rm #2}\rm\dotfill}
% \pdfbookmark[0]{\thecontribCounter~ #3}{#5}
% \pagestyle{fancy}
% \lhead[\fancyplain{}{\thepage}]{\fancyplain{}{\thecontribCounter~ #3}}
% \rhead[\fancyplain{}{#4}]{\fancyplain{}{\thepage}}
% \cfoot{}
% \thispagestyle{empty}
% \hypertarget{#5}{}
% }
%old parameters: 1:toc-title; 2:toc-authors; 3:running-title; 4:running-authors; 5:marker
%changed to %1: toc-title, 2: authors, 3: runningtitle
%Big Command, for inclusion of contributions, doing all sorts of peripheric stuff, also correct klick and point
\renewcommand{\sheader}[3]{
\cleardoublepage
\addtocontents{toc}{\protect\contentsline{section}{\protect\numberline{}%
{{\bfseries #2}\newline{\it #1}\rm\dotfill\ }}{\rm \pageref*{#2}}{#2}}
\hypertarget{#2}{}
\label{#2}
\pdfbookmark[0]{~~~#2}{#2}

\pagestyle{fancy}
\lhead[\fancyplain{}{\small \thepage \hspace{0.5em} #2}]{\fancyplain{}{}}
\rhead[\fancyplain{}{}]{\fancyplain{}{\small #3 \hspace{0.4em} \thepage}}
\cfoot{}
\thispagestyle{fancyfirst}
}



%%%% 
%%%% CW: Version of \sheader for names that can not be used as labels, e.g.
%%%% with \ss
%%%%
\newcommand{\sheaderxlabel}[4]{
\cleardoublepage
\addtocontents{toc}{\protect\contentsline{section}{\protect\numberline{}%
{{\bfseries #2}\newline{\it #1}\rm\dotfill\ }}{\rm \pageref*{#3}}{#3}}
\hypertarget{#3}{}
\label{#3}
\pdfbookmark[0]{~~~#2}{#3}

\pagestyle{fancy}
\lhead[\fancyplain{}{\small \thepage \hspace{0.5em} #2}]{\fancyplain{}{}}
\rhead[\fancyplain{}{}]{\fancyplain{}{\small #4 \hspace{0.4em} \thepage}}
% \lhead[\fancyplain{}{\thepage}]{\fancyplain{}{#4}}
% \rhead[\fancyplain{}{#2}]{\fancyplain{}{\thepage}}
\cfoot{}
\thispagestyle{fancyfirst}
}

%%%% Vorwort... ist sowieso anders Sat Aug 18 10:43:01 2012
%%%% 
%%%% CW: For some reason the some files need the offset while others
%%%% do not Wed May 25 17:17:02 2012
%%%% 
\newcommand{\pdffile}[1]{%
\includepdf[pages=1-,noautoscale=true,pagecommand={},offset=7pt 42pt]{#1}%
}

\newcommand{\pdffilex}[1]{%
\includepdf[pages=1-,noautoscale=true,pagecommand={},offset=0pt 0pt]{#1}%
}

\newcommand{\pdffilexbrill}[1]{%
\includepdf[pages=1-,noautoscale=true,pagecommand={},offset=0pt -23pt]{#1}%
}

\newcommand{\pdffilexschaudt}[1]{%
\includepdf[pages=1-,noautoscale=true,pagecommand={},offset=15pt -25pt]{#1}%
}

\newcommand{\pdffilexlow}[1]{%
\includepdf[pages=1-,noautoscale=true,pagecommand={},offset=0pt -20pt]{#1}%
}

% %%%% 
% %%%% CW: alles etwas hoeher, aehnlich wie im Vorwort
% %%%% Fri Aug 17 09:12:19 2012
% %%%% 
% \newcommand{\pdffile}[1]{%
% \includepdf[pages=1-,noautoscale=true,pagecommand={},offset=5pt 10pt]{#1}%
% }
% 
% \newcommand{\pdffilex}[1]{%
% \includepdf[pages=1-,noautoscale=true,pagecommand={},offset=0pt 10pt]{#1}%
% }
% 
% \newcommand{\pdffilexlow}[1]{%
% \includepdf[pages=1-,noautoscale=true,pagecommand={},offset=0pt -8pt]{#1}%
% }

% ============================================================================

\begin{document}
% \selectlanguage{deutsch}
% \selectlanguage{ngerman}

% %%%%%%%%%%%%%%%%%%%%%%%%%%%%%%%%%%%%%%%%%%%%%%%%%%%%%%%%%%%%%%%%%%%%%%%%%%%%

% {\Large \textbf{Ausgezeichnete Informatikdissertationen 2018}}\\[1ex]
%  {\Large \textbf{Entwurfsfassung vom 22.5.19}}\\[4ex]
%  {\Large \textbf{http://www.wv.inf.tu-dresden.de/GIDiss/}}
% \newpage
% \ \newpage
% \ 

% %%%%%%%%%%%%%%%%%%%%%%%%%%%%%%%%%%%%%%%%%%%%%%%%%%%%%%%%%%%%%%%%%%%%%%%%%%%%

\setcounter{page}{3}
\pagestyle{fancy}
\lhead[\fancyplain{}{\small\thepage}]{\fancyplain{}}
\rhead[\fancyplain{}]{\fancyplain{}{\small\thepage}}
\cfoot{}
% \vspace*{.5ex}


\section*{Vorwort}
\pdfbookmark[1]{Vorwort}{Vorwort}

\newcommand{\Zitat}[1]{\glqq#1\grqq}


Die Gesellschaft f{\"u}r Informatik e.V.\ (GI) vergibt gemeinsam mit der Schweizer
Informatik Gesellschaft (SI) und der {\"O}sterreichischen Computergesellschaft (OCG) j{\"a}hrlich einen Preis f{\"u}r eine
hervorragende Dissertation im Bereich der Informatik. Hierzu z{\"a}hlen nicht nur
Arbeiten, die einen Fortschritt in der Informatik bedeuten, sondern auch
Arbeiten aus dem Bereich der Anwendungen in anderen Disziplinen und Arbeiten,
die die Wechselwirkungen zwischen Informatik und Gesellschaft untersuchen. Die
Auswahl dieser Dissertationen st{\"u}tzt sich auf die von den Universit{\"a}ten und
Hochschulen f{\"u}r diesen Preis vorgeschlagenen Dissertationen. Jede dieser
Hochschulen kann jedes Jahr nur eine Dissertation vorschlagen. Somit sind die
im Auswahlverfahren vorgeschlagenen Kandidatinnen und Kandidaten bereits
\Zitat{Preistr{\"a}ger} ihrer Hochschule.

Die 27 Einreichungen zum Dissertationspreis 2018 belegen die
Bedeutung und auch die Bekanntheit des Dissertationspreises. Wie jedes
Jahr wurden die vorgeschlagenen Arbeiten im Rahmen eines Kolloquiums
im Leibniz-Zentrum f{\"u}r Informatik Schloss Dagstuhl von den Nominierten
vorgestellt. F{\"u}r die Mitglieder des Nominierungsausschusses war das
pers{\"o}nliche Zusammentreffen mit den Nominierten der H{\"o}hepunkt der
Auswahlarbeit, und f{\"u}r die Nominierten hat das Kolloquium sicher eine
Reihe neuer Erfahrungen und wissenschaftlicher Kontakte geboten. Das
wissenschaftlich sehr hohe Niveau der Vortr{\"a}ge, die regen Diskussionen
und die angenehme Atmosph{\"a}re in Schloss Dagstuhl wurde von allen
Teilnehmerinnen und Teilnehmern des Kolloquiums sehr begr{\"u}{\ss}t.

Wie in jedem Jahr fiel es dem Nominierungsausschuss sehr schwer, eine Dissertation auszuw{\"a}hlen, die durch den Preis besonders gew{\"u}rdigt wird. Mit
der Pr{\"a}sentation aller vorgeschlagenen Dissertationen in diesem Band wird die
Ungerechtigkeit, eine aus mehreren ebenb{\"u}rtigen Dissertationen hervorzuheben,
etwas ausgeglichen. Dieser Band soll zudem einen Beitrag zum Wissenstransfer
innerhalb der Informatik und von den Universit{\"a}ten und Hochschulen in die
Bereiche Technik, Wirtschaft und Gesellschaft leisten.

Die beteiligten Gesellschaften zeichnen Herrn  Dr.~Yannic Maus f{\"u}r seine Dissertation \Zitat{The Power of Locality: Exploring the Limits of Randomnesss in Distributed Computing} mit dem Dissertationspreis 2018 aus. 


Im Zentrum der Arbeit von Herr Maus steht die Frage, warum die Laufzeiten der schnellsten randomisierten Algorithmen in verteilten Systemen exponentiell schneller sind als die der schnellsten deterministische Algorithmen. Er hat dazu neue Klassen und Techniken eingef\"uhrt, die einen signifikanten Fortschritt hin zur Beantwortung der Frage darstellen.

Mit dieser Preisverleihung w{\"u}rdigen die beteiligten Gesellschaften -- die
Gesellschaft f{\"u}r Informatik e.V.\ (GI), die Schweizer Informatik Gesellschaft
(SI) und die {\"O}sterreichische Computergesellschaft (OCG) -- eine herausragende theoretische Arbeit, mit deren Hilfe randomisierte Algorithmen in verteilten Systemen deutlich besser verstanden werden.\bigskip

Ein besonderer Dank gilt dem Nominierungsausschuss, der sehr effizient
und konstruktiv zusammengearbeitet hat. Bei Frau Emmanuelle-Anna Dietz Saldanha
m{\"o}chte ich mich f{\"u}r die Unterst{\"u}tzung bei der Entgegennahme der
vorgeschlagenen Dissertationen, f{\"u}r die Organisation des Kolloquiums
sowie f{\"u}r die Zusammenstellung und Anpassung der Beitr{\"a}ge an das
Format der GI-Edition Lecture Notes in Informatik (LNI) bedanken. F{\"u}r
die finanzielle Unterst{\"u}tzung des Nominierungskolloquiums sei den
beteiligten Gesellschaften gedankt. Die Gastfreundlichkeit und die
hervorragende Bewirtung in Dagstuhl trugen zum Erfolg des Kolloquiums
bei, wof{\"u}r ich mich an dieser Stelle ebenfalls herzlich bedanke.

\bigskip

\noindent
\noindent
Steffen H{\"o}lldobler\\
Dresden im August  2019

\vfill

\centerline{\includegraphics[width=0.85\textwidth]{gidiss18gruppe.jpg}}

% \newpage
% 
% % %%

\newpage

% \setcounter{page}{5}
\section*{Kandidat$^*$innen f\"ur den GI-Dissertationspreis 2018}

\setlength{\tabcolsep}{0pt}

\begin{tabular}{l@{\hspace{10pt}}l}
Dr.-Ing.\ Arriel, Juliana &	Otto-von-Guericke-Universit{\"a}t Magdeburg\\
Dr.\ Berrang, Pascal &	Universit{\"a}t des Saarlandes\\
Dr. techn.\ B{\"o}hmer, Kristof & Universit{\"a}t Wien / Fakult{\"a}t f{\"u}r Informatik\\
Dr. rer. nat.\ Bj{\"o}rkelund, Anders &	Universit{\"a}t Stuttgart\\
Dr.\ Bloessl, Bastian &	Universit{\"a}t Paderborn\\
Dr. rer. nat.\ Brachmann, Eric &	Technische Universit{\"a}t Dresden\\
Dr. rer. nat.\ Buschek, Daniel &	Ludwig-Maximilians-Universit{\"a}t\\
Dr.\ D{\"u}hrkop, Kai &	Friedrich-Schiller-Universit{\"a}t Jena\\
Dipl.-Ing. BSc\ Eichlseder, Maria &	Technische Universit{\"a}t Graz\\
Dr.\ Faessler, Matthias &	Universit{\"a}t Z{\"u}rich\\
Dr.\ Fr{\"o}mmgen, Alexander &	Technische Universit{\"a}t Darmstadt\\
Dr. rer. nat.\ Herbst, Nikolas &	Julius-Maximilians-Universit{\"a}t W{\"u}rzburg\\
Dr. rer. nat.\ Immler, Fabian &	Technische Universit{\"a}t M{\"u}nchen\\
Dr.-Ing.\ Kr{\"u}ger, Julia &	Universit{\"a}t zu L{\"u}beck\\
Dr. rer. nat.\ Maus, Yannic &	Albert-Ludwigs-Universit{\"a}t Freiburg\\
Dr.\ Maystre, Lucas &	EPFL - Ecole Polytechnique Federale de Lausanne\\
Dr. rer. nat.\ Nicolaescu, Ana &	RWTH Aachen University\\
Dr.-Ing.\ Pfaff, Florian &	Karlsruher Institut f{\"u}r Technologie\\
Dr.\ Piatkowski, Nico Philipp &	TU Dortmund\\
Dr.\ Rehr, Robert &	Universit{\"a}t Hamburg\\
DI Dr.\ Rigger, Manuel &	Johannes Kepler Universit{\"a}t Linz\\
Dr. rer. nat.\ Sacha, Dominik &	Universit{\"a}t Konstanz\\
Dr.\ Schlachter, Uli Christian &	Carl von Ossietzky Universit{\"a}t Oldenburg\\
Dr.\ Sch{\"u}tt, Kristof &	Technische Universit{\"a}t Berlin\\
Dr.\ Shirinzadeh, Saeideh &	Universit{\"a}t Bremen\\
Dr.\ techn.\ Spiel, Katta &	TU Wien\\
Dr.\ Urbat, Henning &	TU Braunschweig\\
Dr.\ van der Heijden, Rens &	Universit{\"a}t Ulm\\

\end{tabular}


%\vspace{8mm} 
\newpage
\section*{Mitglieder des Nominierungsausschusses\\ f\"ur den GI-Dissertationspreis 2018}
% ****

\centerline{\includegraphics[width=\textwidth]{gidiss18jury}}
\vspace{4pt}
Von links nach rechts:
\vspace{2pt}

\begin{tabular}{p{6.3cm}l}
Prof.\ Dr.\ R{\"u}diger Reischuk & Universit{\"a}t zu L{\"u}beck \\
Prof.\ Dr.-Ing.\ Felix Freiling & Universit{\"a}t Erlangen-N{\"u}rnberg \\
Prof.\ Dr.\ Bj{\"o}rn Scheuermann &  Humbold-Universit\"at zu Berlin \\
Prof.\ Dr.\ Abraham Bernstein & Universit\"at Z\"urich\\ 
Prof.\ Dr.\ Steffen H\"olldobler (Vorsitzender) & Technische Universit\"at Dresden\\
Prof.\ Dr.\ Gustaf Neumann &   Wirtschaftsuniversit\"at Wien\\
Prof.\ Dr.\ Hans-Peter Lenhof & Universit{\"a}t des Saarlandes\\

\end{tabular}


\vspace{6pt}
Nicht im Bild:
\vspace{2pt}

\begin{tabular}{p{6.2cm}l}
Prof.\ Dr.\ Sven Apel &  Universit{\"a}t des Saarlandes \\
Prof.\ Dr.\ Paul Molitor & Martin-Luther-Universit{\"a}t Halle-Wittenberg\\
Prof.\ Dr.\ Nicole Schweikardt & Humbold-Universit\"at zu Berlin\\
Prof.\ Dr.\ Myra Spiliopoulou & Otto-von-Guericke-Universit{\"a}t Magdeburg\\
Prof.\ Dr.\ Sabine S{\"u}sstrunk & {\'E}cole Polytechnique F{\'e}d{\'e}rale de Lausanne\\
Prof.\ Dr.\ Klaus Wehrle & RWTH Aachen \\
\end{tabular}


\newpage

\pdfbookmark[1]{Inhalt}{Inhalt}

% \enlargethispage{-1\baselineskip}

%%%%
%%%% CW for 2011: squeeze toc a bit
%%%% 
\setlength{\parskip}{0.1cm}

\tableofcontents

% \newpage

% \end{document}

%offset : 1.: x-Achse(kleiner=links), y-Achse(groesser=oben)
%typisches pstopdf offset=-9pt -25pt meistens falsch ;) 	RICHTIG FUER ODD
%typisches pstopdf offset=9pt -25pt odd even problem ....	RICHTIG FUER EVEN
%offset=0pt -25pt,trim= 18pt 0pt 0pt 0pt			DAS IST DIE LOESUNG! (gammlig)
%typisches doc offset=-6pt 18pt

\sheader 
{Personalized Recommender Systems for Software Product Line Configurations}
% {Personalisierte Recommender Systems f{\"u}r Software-Produktlinienkonfigurationen}
{Arriel, Juliana}
{Personalized Recommender Systems for Software Product Line Configurations}
% {Personalisierte Recommender Systems f{\"u}r Software-Produktlinienkonfigurationen}
\pdffile{pdfs/invited_paper_11}


\sheader 
{Quantifying and Mitigating Privacy Risks in Biomedical Data}
% {Quantifizierung und Schutz der Privatsph{\"a}re in der (Epi-)genetik}
{Berrang, Pascal}
{Quantifying and Mitigating Privacy Risks in Biomedical Data}
% {Quantifizierung und Schutz der Privatsph{\"a}re in der (Epi-)genetik}
\pdffile{pdfs/invited_paper_3}


\sheader 
{Behavior Verification for Business Processes \\ based on Testing and Anomaly Detection} %Testing and Anomaly Detection for Business Processes
% {Verhaltensverifizierung f{\"u}r Gesch{\"a}ftsprozesse basierend auf Testverfahren und Anomalieerkennung}
{B{\"o}hmer, Kristof}
{Behavior Verification for Business Processes based on Testing and Anomaly Detection}
% {Verhaltensverifizierung f{\"u}r Gesch{\"a}ftsprozesse basierend auf Testverfahren und Anomalieerkennung}
\pdffile{pdfs/invited_paper_14}

\sheader 
{Online Learning of Latent Linguistic Structure with Approximate Search}
% {Inkrementelles Lernen latenter linguistischer Strukturen durch approximative Suche}
{Bj{\"o}rkelund, Anders}
{Online Learning of Latent Linguistic Structure with Approximate Search}
% {Inkrementelles Lernen latenter linguistischer Strukturen durch approximative Suche}
\pdffile{pdfs/invited_paper_20}


\sheader 
{A Physical Layer Experimentation Framework for Automotive WLAN}
% {Eine Experimentierumgebung zur Studie der physikalischen
% Schicht einer WLAN-Variante für die Anwendung in
% Fahrzeugnetzen}
{Bloessl, Bastian}
% {Eine Experimentierumgebung zur Studie der physikalischen
% Schicht einer WLAN-Variante für die Anwendung in
% Fahrzeugnetzen}
{A Physical Layer Experimentation Framework for Automotive WLAN}
\pdffile{pdfs/invited_paper_18}


\sheader 
{Learning to Predict Dense Correspondences for 6D Pose Estimation}
% {6D Posensch{\"a}tzung mit gelernten, dichten
% Korrespondenzvorhersagen}
{Brachmann, Eric}
{Learning to Predict Dense Correspondences for 6D Pose Estimation}
% {6D Posensch{\"a}tzung mit gelernten, dichten
% Korrespondenzvorhersagen}
\pdffile{pdfs/invited_paper_9}


\sheader 
{Behaviour-Aware Mobile Touch Interfaces}
% {Verhaltenssensitive Nutzerschnittstellen für mobile Ger{\"a}te
% mit ber{\"u}hrungsempfindlichem Bildschirm}
{Buschek, Daniel}
{Behaviour-Aware Mobile Touch Interfaces}
% {Verhaltenssensitive Nutzerschnittstellen für mobile Ger{\"a}te
% mit ber{\"u}hrungsempfindlichem Bildschirm}
\pdffile{pdfs/invited_paper_10}


\sheader 
{Computational Methods for Small Molecule Identification}
% {Die Analyse kleiner Molek{\"u}le mittels Methoden der
% Kombinatorik und des maschinellen Lernens}
{D{\"u}hrkop, Kai}
{Computational Methods for Small Molecule Identification}
% {Die Analyse kleiner Molek{\"u}le mittels Methoden der
% Kombinatorik und des maschinellen Lernens}
\pdffile{pdfs/invited_paper_7}


\sheader 
{Differential Cryptanalysis of Symmetric Primitives}
{Eichlseder, Maria}
{Differential Cryptanalysis of Symmetric Primitives}
\pdffile{pdfs/invited_paper_22}


\sheader 
{Quadrator Control for Accurate Agile Flight}
{Faessler, Matthias}
{Quadrator Control for Accurate Agile Flight}
% \pdffile{pdfs/invited_paper_}


\sheader 
{Programming Models and Extensive Evaluation Support\\ for MPTCP Scheduling, Adaptation Decisions, and DASH Video Streaming} % zu lang
{Fr{\"o}mmmgen, Alexander}
{Programmiermodelle \& Unterstützung f{\"u}r umfassende Evaluationen für Kommunikationssyst}
\pdffile{pdfs/invited_paper_13}


\sheader 
{Methods and Benchmarks for Auto-Scaling Mechanisms\\ in Elastic Cloud Environments}
{Herbst, Nikolas}
{Methods and Benchmarks for Auto-Scaling Mechanisms in Elastic Cloud Environments}
\pdffile{pdfs/invited_paper_6}


\sheader 
{A Verified ODE Solver and Smale's 14th Problem}
{Immler, Fabian}
{A Verified ODE Solver and Smale's 14th Problem}
\pdffile{pdfs/invited_paper_24}


\sheader 
{Statistical appearance models based \\on probabilistic correspondences for medical image analysis} % zu lang
% {Statistische Appearance-Modelle basierend auf probabilistischen Korrespondenzen f{\"u}r die medizinische Bildanalyse}
{Kr{\"u}ger, Julia}
{Probabilistische Appearance-Modelle f{\"u}r die medizinische Bildanalyse}
% {Statistische Appearance-Modelle basierend auf probabilistischen Korrespondenzen f{\"u}r die medizinische Bildanalyse}
\pdffile{pdfs/invited_paper_26}


\sheader 
{The Power of Locality Exploring the Limits \\of Randomness in Distributed Computing}
{Maus, Yannic}
{The Power of Locality Exploring the Limits of Randomness in Distributed Computing}
\pdffile{pdfs/invited_paper_12}


\sheader 
{Efficient Learning from Comparisons}
{Maystre, Lucas}
{Efficient Learning from Comparisons}
\pdffile{pdfs/invited_paper_15}


\sheader 
{Behavior-Based Architecture Conformance Checking}
{Nicolaescu, Ana}
{Behavior-Based Architecture Conformance Checking}
\pdffile{pdfs/invited_paper_17}


\sheader 
{Multitarget Tracking Using Orientation Estimation for Optical Belt Sorting}
{Pfaff, Florian}
{Multitarget Tracking Using Orientation Estimation for Optical Belt Sorting}
\pdffile{pdfs/invited_paper_16}


\sheader 
{Exponential families on resource-constrained systems}
{Piatkowski, Nico Philipp}
{Exponential families on resource-constrained systems}
\pdffile{pdfs/invited_paper_27}


\sheader 
{Robust Speech Enhancement Using\\ Statistical Signal Processing and Machine-Learning}
{Rehr, Robert}
{Robust Speech Enhancement Using Statistical Signal Processing and Machine-Learning}
\pdffile{pdfs/invited_paper_21}


\sheader 
{Memory-safe Execution of Low-level Languages on a Java Virtual Machine}
{Rigger, Manuel}
{Memory-safe Execution of Low-level Languages on a Java Virtual Machine}
\pdffile{pdfs/invited_paper_8}


\sheader 
{Knowledge Generation in Visual Analytics: \\Integrating Human and Machine Intelligence for Exploration of Big Data} % zu lang
{Sacha, Dominik}
{Wissensbildung in der visuellen Datenanalyse}
\pdffile{pdfs/invited_paper_2}


\sheader 
{Petri Net Synthesis and Modal Specifications}
{Schlachter, Uli Christian}
{Petri Net Synthesis and Modal Specifications}
\pdffile{pdfs/invited_paper_25}


\sheader 
{Learning Representations of Atomistic Systems with Deep Neural Networks}
{Sch{\"u}tt, Kristof}
{Learning Representations of Atomistic Systems with Deep Neural Networks}
\pdffile{pdfs/invited_paper_4}


\sheader 
{Synthesis and Optimization \\for Logic-in-Memory Computing using Memristive Devices}
{Shirinzadeh, Saeideh}
{Synthesis and Optimization for Logic-in-Memory Computing using Memristive Devices}
\pdffile{pdfs/invited_paper_1}


\sheader 
{Evaluating Experiences of Autistic Children with Technologies in Co-Design}
{Spiel, Katta}
{Evaluating Experiences of Autistic Children with Technologies in Co-Design}
\pdffile{pdfs/invited_paper_19}


\sheader 
{A Categorical Approach to Algebraic}
{Urbat, Henning}
{A Categorical Approach to Algebraic}
\pdffile{pdfs/invited_paper_23}


\sheader 
{Misbehavior Detection in Cooperative Intelligent Transport Systems}
{van der Heijden, Rens}
{Misbehavior Detection in Cooperative Intelligent Transport Systems}
\pdffile{pdfs/invited_paper_5}

\end{document}
